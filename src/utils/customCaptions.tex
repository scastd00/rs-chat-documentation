\usepackage{newfloat}

% placement = H means that the float environment will be placed where the \begin{} command is, strictly.

\newlistof{codeBlock}{cod}{Índice de bloques de código}
\DeclareFloatingEnvironment[
	fileext = cod,
	listname = {Índice de bloques de código},
	name = Código,
	placement = H,
]{codeBlock}

\newlistof{umlDiagram}{dia}{Índice de diagramas UML}
\DeclareFloatingEnvironment[
	fileext = dia,
	listname = {Índice de diagramas UML},
	name = Diagrama UML,
	placement = H
% See docs -> https://www.overleaf.com/learn/latex/Positioning_images_and_tables#The_figure_environment
]{umlDiagram}


\renewcommand{\cftcodeBlockpresnum}{Código~}
% leave some space for the title of the code block
\setlength{\cftcodeBlocknumwidth}{5.5em}

\renewcommand{\cftumlDiagrampresnum}{Diagrama UML~}
% leave some space for the title of the diagram
\setlength{\cftumlDiagramnumwidth}{9.5em}
