\usepackage{newfloat}

% placement = H means that the float environment will be placed where the \begin{} command is, estrictly.

\DeclareFloatingEnvironment[
	fileext = cod,
	listname = {Índice de bloques de código},
	name = Código,
	placement = H,
]{codeBlock}

\DeclareFloatingEnvironment[
	fileext = dia,
	listname = {Índice de diagramas UML},
	name = Diagrama UML,
	placement = H % See docs -> https://www.overleaf.com/learn/latex/Positioning_images_and_tables#The_figure_environment
]{umlDiagram}
