\thispagestyle{empty}

\begin{table}[ht]
	\centering
	\label{tab:info}
	\begin{tabularx}{\textwidth}{|X|X|}
		\hline

		\begin{center}
			\LARGE\textbf{UNIVERSIDAD DE LEÓN} \\[4pt]
			\LARGE\textbf{Escuela de Ingenierías Industrial, Informática y} \\
			\LARGE\textbf{Aeroespacial} \\[16pt]
			\LARGE\textbf{GRADO EN INGENIERÍA INFORMÁTICA} \\[4pt]
			\LARGE\textbf{Trabajo de Fin de Grado} \\[35pt]
		\end{center}
		\\

		\hline

		\\[-6.6ex]
		\begin{flushleft}
			\textbf{ALUMNO:} Samuel Castrillo Domínguez
		\end{flushleft}
		\\[-3ex]

		\hline

		\\[-6.6ex]
		\begin{flushleft}
			\textbf{TUTOR:} Eva María Cuervo Fernández
		\end{flushleft}
		\\[-3ex]

		\hline

		\\[-7ex]
		\begin{flushleft}
			\textbf{TÍTULO:} RS Chat: Aplicación web de chat para comunicación en tiempo real entre estudiantes y
			docentes.
		\end{flushleft}
		\\[-3ex]

		\hline

		\\[-6.6ex]
		\begin{flushleft}
			\textbf{TITLE:} RS Chat: Real time chat web application for students and teachers communication.
		\end{flushleft}
		\\[-3ex]

		\hline

		\\[-6.6ex]
		\begin{flushleft}
			\textbf{CONVOCATORIA:} Julio, 2023
		\end{flushleft}
		\\[-3ex]

		\hline

		\\[-6.6ex]
		\begin{flushleft}
			\textbf{RESUMEN:} \\
			{RS Chat es una aplicación web para la interacción entre estudiantes y docentes
			en tiempo real. Su objetivo es facilitar la comunicación y colaboración
			entre los usuarios. Ofrece un sistema de almacenamiento del historial
			de mensajes, notificaciones y compartición de contenido multimedia, filtrando
			las imágenes inapropiadas debido al entorno educativo al que se ha dirigido.}
		\end{flushleft}
		\\[-1.5ex]

		\hline

		\\[-6.6ex]
		\begin{flushleft}
			\textbf{ABSTRACT:} \\
			{RS Chat is a web application for real time interaction between students and
			teachers. Its objective is to facilitate communication and collaboration
			between users. It offers a message history storage system, notifications and
			sharing of multimedia content, filtering inappropriate images due to the
			educational environment to which it has been directed.}
		\end{flushleft}
		\\[-1.5ex]

		\hline

		\\[-6.8ex]
		\begin{flushleft}
			\textbf{Palabras clave:} chat, tiempo real, estudiantes, docentes, comunicación, multimedia, inteligencia
			artificial, filtrado de imágenes.
		\end{flushleft}
		\\[-3ex]

		\hline

		\begin{tabular}{p{0.45\textwidth}|p{0.5\textwidth}}
			\textbf{Firma del alumno:}
			&
			\textbf{VºBº Tutor/es:} \\

			% Leave some blank space
			{} & {} \\[10ex]
		\end{tabular}
		\\

		\hline
	\end{tabularx}
\end{table}

