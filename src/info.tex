\thispagestyle{empty}

\begin{table}[ht]
	\centering
	\label{tab:info}
	\begin{tabularx}{\textwidth}{|X|X|}
		\hline

		\begin{center}
			\LARGE\textbf{UNIVERSIDAD DE LEÓN} \\[4pt]
			\LARGE\textbf{Escuela de Ingenierías Industrial, Informática y} \\
			\LARGE\textbf{Aeroespacial} \\[16pt]
			\LARGE\textbf{GRADO EN INGENIERÍA INFORMÁTICA} \\[4pt]
			\LARGE\textbf{Trabajo de Fin de Grado} \\[35pt]
		\end{center}
		\\

		\hline

		\\[-6.6ex]
		\begin{flushleft}
			\textbf{ALUMNO:} Nombre y apellidos del alumno
		\end{flushleft}
		\\[-3ex]

		\hline

		\\[-6.6ex]
		\begin{flushleft}
			\textbf{TUTOR:} Nombre y apellidos del tutor
		\end{flushleft}
		\\[-3ex]

		\hline

		\\[-7ex]
		\begin{flushleft}
			\textbf{TÍTULO:} Título del trabajo
		\end{flushleft}
		\\[-3ex]

		\hline

		\\[-6.6ex]
		\begin{flushleft}
			\textbf{TITLE:} Title of the work
		\end{flushleft}
		\\[-3ex]

		\hline

		\\[-6.6ex]
		\begin{flushleft}
			\textbf{CONVOCATORIA:} Mes, Año
		\end{flushleft}
		\\[-3ex]

		\hline

		\\[-6.6ex]
		\begin{flushleft}
			\textbf{RESUMEN:} \\
			{El resumen reflejará las ideas principales de cada una de las partes del
			trabajo, pudiendo incluir un avance de los resultados obtenidos. Constará de
			un único párrafo y se recomienda una longitud no superior a 300 palabras. En
			cualquier caso, no deberá superar una página de longitud.}
		\end{flushleft}
		\\[-1.5ex]

		\hline

		\\[-6.6ex]
		\begin{flushleft}
			\textbf{ABSTRACT:} \\
			{Abstract will reflect the main ideas of each part of the work, including
			an advance of the results obtained. It will consist of a single paragraph and
			it is recommended a length not superior to 300 words. In any case, it should
			not exceed a page of length.}
		\end{flushleft}
		\\[-1.5ex]

		\hline

		\\[-6.8ex]
		\begin{flushleft}
			\textbf{Palabras clave:} Lorem, ipsum, dolor, sit, amet.
		\end{flushleft}
		\\[-3ex]

		\hline

		\begin{tabular}{p{0.45\textwidth}|p{0.5\textwidth}}
			\textbf{Firma del alumno:}
			&
			\textbf{VºBº Tutor/es:} \\

			% Leave some blank space
			{} & {} \\[10ex]
		\end{tabular}
		\\

		\hline
	\end{tabularx}
\end{table}

