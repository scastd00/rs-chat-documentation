\sect{Objetivos}{objectives}
En un principio, el objetivo de la aplicación era su integración de forma completa con la plataforma de
\textit{Moodle} de la Universidad de León, para que tanto estudiantes y profesores pudieran acceder a ella.
Sin embargo, debido a que era una idea demasiado ambiciosa, se ha optado por generalizarlo más a un chat de
mensajes instantáneos para que cualquier persona con acceso a internet pueda utilizarlo (pero manteniendo la temática
de estudiantes/profesores, en caso de llegarse a utilizar en un futuro).
Actualmente, en cualquier aplicación de chat, se permite compartir archivos, imágenes, vídeos, etc.\ con los demás
usuarios.

Se podrán tener varios chats abiertos a la vez, pudiendo cambiar entre ellos fácilmente (teniendo una disposición en
pestañas, mostrando un pequeño icono con el número de mensajes que no han sido leídos todavía).
Además, se podrán crear grupos de chat para que varias personas puedan
comunicarse entre sí, pudiendo compartir un código de invitación para que otros usuarios se unan al grupo en específico.
Estos canales podrán ser públicos o privados, pudiendo ser generados por cualquier usuario registrado en la
aplicación.
Los canales privados solo podrán ser vistos por los usuarios que hayan sido invitados a ellos.
Los canales públicos podrán ser vistos por cualquier persona que tenga acceso a la aplicación, sin necesidad de
invitación.
También se podrán enviar mensajes a una sola persona, pudiendo tener una conversación privada sin notificar al resto
de usuarios.
