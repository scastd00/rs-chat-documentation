\sect{Objetivos}{objectives}
En un principio, el objetivo principal de la aplicación era su integración de forma completa con la plataforma de
\textit{Moodle} de la Universidad de León (como un complemento a los foros), para que tanto estudiantes como profesores
pudieran emplearla e interactuar de manera más rápida.\ Sin embargo, debido a que era una idea demasiado ambiciosa,
se ha optado por generalizarlo más a un chat de mensajes instantáneos para que cualquier persona con acceso a
Internet pueda utilizarlo (pero manteniendo la temática de estudiantes/profesores, en caso de llegarse a utilizar en
un futuro).

Actualmente, en cualquier aplicación de chat, se permite compartir archivos, imágenes, vídeos, etc.\ con los demás
usuarios.
Se pueden tener varios chats abiertos a la vez, pudiendo cambiar entre ellos fácilmente (teniendo una disposición en
pestañas, mostrando un pequeño icono con el número de mensajes que no han sido leídos todavía).
Además, se permite a los usuarios crear grupos de chat para que varias personas puedan
comunicarse entre sí, pudiendo compartir un código de invitación para que otros usuarios se unan a ese grupo.
Estos canales podrán ser \boldFont{públicos} o \boldFont{privados}.
Los canales \boldFont{privados} solo pueden ser vistos por los usuarios que hayan sido invitados a ellos, de manera
explícita o con el código de invitación.\ Los canales \boldFont{públicos} son accesibles para cualquier persona que
tenga acceso a la aplicación, sin necesidad de invitación.\ También se pueden enviar mensajes a una sola persona,
estableciendo una conversación privada sin notificar al resto de usuarios conectados.

Una de las características que no ofrece ninguna aplicación de chat es la posibilidad de
\boldFont{prevenir el envío de imágenes inapropiadas}, por lo que este es uno de los objetivos principales de la
aplicación.\ Además de esto, el resto de los objetivos para el proyecto son los siguientes:

\begin{itemize}
	\item Desarrollar una aplicación web para la comunicación a través de chat entre los usuarios.
	\item Se debe ofrecer un sistema de grupos y chats individuales para los usuarios.
	\item Los usuarios deben poder intercambiar archivos entre ellos.
	\item Garantizar el debido cumplimiento de la política de privacidad de los usuarios
	(ver sección~\ref{sec:privacidad} para más información).
	\item Establecer un sistema de roles, para permitir la interacción entre usuarios de manera
	efectiva y fluida.\ Esto les ayuda a determinar a qué personas comunicar incidencias u otra
	información.
	\item Los administradores deben tener un panel de control para gestionar todo lo relacionado con la aplicación
	y el uso de recursos por parte del servidor, para evitar sobrecargas o caídas.
	\item Ofrecer un sistema de recompensa a los usuarios que más utilicen la aplicación, otorgando insignias
	que se mostrarán en su perfil.
	\item Permitir a los usuarios visualizar las estadísticas de mensajes enviados, archivos compartidos, etc.\ en
	la aplicación.
\end{itemize}
\label{itm:alcance_objetivos}
