\sect{Tecnologías y herramientas}{technologies}
A continuación, se presenta una breve descripción de las tecnologías y herramientas más importantes entre todas las que
se han utilizado para la elaboración de este trabajo:

\begin{itemize}
	\item \boldFont{Java}: es el lenguaje de programación utilizado para la implementación de la parte backend de la
	aplicación.
	\item \boldFont{Spring Boot}: es un framework de Java que permite la creación de aplicaciones web.
	\item \boldFont{Maven}: es un gestor de dependencias de Java que facilita la descarga e integración de las
	librerías utilizadas en la aplicación.\ El fichero \monoFont{pom.xml} tiene una etiqueta concreta donde se añaden
	todas las dependencias a utilizar, proporcionando su nombre, versión e identificador.
	\item \boldFont{React}: es una biblioteca de JavaScript que permite la creación de interfaces de usuario mediante
	componentes.
	\item \boldFont{Vercel}: es un servicio de hosting que permite el despliegue de la parte frontend de la
	aplicación web.
	\item \boldFont{Git}: es un sistema de control de versiones que ha facilitado el desarrollo de la aplicación desde
	diferentes ordenadores.
	\item \boldFont{GitHub}: es una plataforma que permite el almacenamiento de repositorios de Git.
	\item \boldFont{IntelliJ IDEA}: es el IDE que ha permitido la implementación de la aplicación de forma completa.
	Se ha utilizado para la creación de los proyectos de frontend y backend, así como para la
	elaboración de este documento mediante el uso de \LaTeX.
\end{itemize}
