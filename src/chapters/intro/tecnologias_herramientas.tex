\sect{Tecnologías y herramientas}{technologies}
A continuación, se presenta una breve descripción de las tecnologías y herramientas más importantes entre todas las que
se han utilizado para la elaboración de este trabajo:

\begin{itemize}
	\item Java: es el lenguaje de programación utilizado para la implementación del backend de la aplicación.
	\item Spring Boot: es un framework de Java que permite la creación de aplicaciones web.
	\item React: es una biblioteca de JavaScript que permite la creación de interfaces de usuario mediante componentes.
	\item Vercel: es un servicio de hosting que permite el despliegue de la parte de frontend de la aplicación web.
	\item Git: es un sistema de control de versiones que ha facilitado el desarrollo de la aplicación desde diferentes
	ordenadores.
	\item GitHub: es una plataforma que permite el almacenamiento de repositorios de Git.
	\item IntelliJ IDEA: es el IDE que ha permitido la implementación de la aplicación de forma completa.
	Se ha utilizado para la creación de los proyectos de frontend y backend, así como para la
	elaboración de este documento mediante el uso de \LaTeX.
	\item Maven: es un gestor de dependencias que facilita la descarga e integración de las librerías utilizadas en la
	aplicación.\ El fichero \monoFont{pom.xml} tiene una etiqueta donde se añaden todas las dependencias a utilizar,
	proporcionando su nombre, versión e identificador.
\end{itemize}
