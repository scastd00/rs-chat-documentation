\sect{Gestión de riesgos}{riesgos}

La gestión de riesgos es un proceso iterativo que comienza en la fase de planificación del proyecto y continúa a lo
largo de todo el ciclo de vida del mismo.\ El objetivo de este proceso es identificar, analizar y responder a los
riesgos del proyecto.\ Para ello, se han seguido los siguientes pasos:

\begin{enumerate}
	\item Identificación de riesgos: se identifican los riesgos que pueden afectar al proyecto.
	\item Análisis de riesgos: se analizan los riesgos identificados para determinar su probabilidad de ocurrencia y su
	impacto en el proyecto.
	\item Planificación de respuestas a los riesgos: se planifican las respuestas a los riesgos identificados.
	\item Seguimiento y control de riesgos: se monitorizan los riesgos identificados y se identifican nuevos riesgos.
\end{enumerate}
\label{itm:riesgos_pasos}

\subsect{Identificación de riesgos}{riesgos_identificacion}

% Todo: Añadir riesgos de la planificación del proyecto
