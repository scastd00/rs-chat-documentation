\sect{Gestión de riesgos}{riesgos}

La gestión de riesgos es un proceso iterativo que comienza en la fase de planificación del proyecto y continúa a lo
largo de todo el ciclo de vida del mismo~\cite{riesgos_agile}.\ El objetivo de este proceso es identificar, analizar
y responder a los riesgos del proyecto, para evitar que afecten al mismo de manera negativa.\ Para ello, se han seguido
los siguientes pasos:

\begin{enumerate}
	\item Identificación de riesgos: se identifican los riesgos que pueden afectar al proyecto.
	\item Análisis de riesgos: se analizan los riesgos identificados para determinar su probabilidad de ocurrencia y su
	impacto en el proyecto.
	\item Planificación de respuestas a los riesgos: se planifican las respuestas a los riesgos identificados.
	\item Seguimiento y control de riesgos: se monitorizan los riesgos identificados y se identifican nuevos riesgos.
\end{enumerate}
\label{itm:riesgos_pasos}

A continuación, se detallan los riesgos identificados en el proyecto, así como su probabilidad de ocurrencia e impacto
en el mismo.

\begin{table}[H]
	\centering
	\caption{Riesgos identificados en el proyecto. (Fuente: Elaboración propia).}
	\begin{tabular}{lcc}
		\toprule
		\textbf{Riesgo}           & \textbf{Probabilidad} & \textbf{Impacto} \\
		\midrule
		Errores de estimación       & Alta                  & Muy alto         \\
		Falta de recursos           & Baja                  & Alto             \\
		Cambios en los requisitos   & Media                 & Medio            \\
		Incompatibilidades software & Media                 & Medio            \\
		Cambios en las tecnologías  & Muy baja              & Muy alto         \\
		\bottomrule
	\end{tabular}
	\label{tab:riesgos_identificados}
\end{table}

Como en este proyecto se emplean metodologías ágiles, en cada iteración se puede reaccionar a los cambios o riesgos
que surjan de una manera más sencilla que con las metodologías tradicionales~\cite{Lagestion42:online}.
