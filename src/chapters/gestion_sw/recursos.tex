\sect{Gestión de recursos}{recursos}

En este capítulo se describen los recursos humanos y materiales que se han necesitado para el desarrollo del proyecto.

\subsect{Especificación de recursos}{especificacion_recursos}

Cuando se habla de recursos humanos, en términos de un proyecto, se hace referencia a las personas que van a
participar en el mismo.\ Los recursos materiales son los equipos y herramientas que se van a utilizar para llevar a
cabo el proyecto.
En el caso de este, que ha sido implementado enteramente por una persona, se simplifican los cálculos, ya que se
conocen de manera más precisa.\ A continuación, se presenta una tabla con la especificación de los recursos para cada
rol que se ha desempeñado en el desarrollo del proyecto:

\begin{table}[H]
	\centering
	\caption{Especificación de recursos y salarios.}
	\begin{tabular}{lcccc}
		\toprule
		\textbf{Recursos} & \textbf{Sal. Anual} & \textbf{Sal. Mensual} & \textbf{Sal. Diario} & \textbf{Sal.
		Hora} \\
		\midrule
		Desarrollador backend  & \EUR{31,412} & \EUR{2,243.71} & \EUR{112.18} & \EUR{14.02} \\
		Desarrollador frontend & \EUR{37,123} & \EUR{2,651.64} & \EUR{132.58} & \EUR{16.57} \\
		Diseñador              & \EUR{22,417} & \EUR{1,601.21} & \EUR{80.06}  & \EUR{10.00} \\
		Tester                 & \EUR{28,036} & \EUR{2,002.57} & \EUR{100.12} & \EUR{12.51} \\
		Técnico de sistemas    & \EUR{24,422} & \EUR{1,744.42} & \EUR{87.22}  & \EUR{10.90} \\
		\bottomrule
	\end{tabular}
	\label{tab:especificacion_recursos}
\end{table}

Se ha considerado que la jornada laboral es de 8 horas diarias, 5 días a la semana, y 14 pagas al año.\ Los datos de
la columna \textit{Sal. anual} se han obtenido del buscador de sueldos de \textit{Indeed}~\cite{SueldosI69:online}
a fecha 8 de junio de 2023\@.\ Esta web se encarga de calcular el sueldo promedio de los trabajos ofrecidos en España
para un puesto en concreto.\ El resto de datos se han calculado con las siguientes fórmulas:

\begin{center}
	\begin{equation}
		\text{Sal. Mensual} = \frac{\text{Sal. Anual}}{14}
		\label{eq:salario_mensual}
	\end{equation}

	\begin{equation}
		\text{Sal. Diario} = \frac{\text{Sal. Mensual}}{20}
		\label{eq:salario_diario}
	\end{equation}

	\begin{equation}
		\text{Sal. Hora} = \frac{\text{Sal. Diario}}{8}
		\label{eq:salario_hora}
	\end{equation}
\end{center}

\subsect{Asignación de recursos}{asignacion_recursos}

Una estimación del trabajo que se ha realizado con el coste asociado se puede ver en la tabla~\ref{tab:coste_recursos}.
Se ha calculado el tiempo que se ha dedicado a cada rol y el coste asociado a cada uno de ellos utilizando los datos de
la pila del producto (ver anexo~\ref{anx:product-backlog-notion}).

\begin{table}[H]
	\centering
	\caption{Coste de los recursos.}
	\begin{tabular}{lccc}
		\toprule
		\textbf{Recursos}    & \textbf{Trabajo (horas)} & \textbf{Coste (\euro)} \\
		\midrule
		Desarrollador backend  & 2000                     & \EUR{28,046.42}        \\
		Desarrollador frontend & 1500                     & \EUR{24,859.15}        \\
		Diseñador              & 300                      & \EUR{3,002.27}         \\
		Tester                 & 500                      & \EUR{6,258.03}         \\
		Técnico de sistemas    & 700                      & \EUR{7,631.87}         \\
		\bottomrule
		Totales                & 5000                     & \EUR{69,797.74}        \\
	\end{tabular}
	\label{tab:coste_recursos}
\end{table}

Los costes del material necesario para la realización del proyecto se han dividido en dos categorías: hardware y
software.\ En las siguientes dos tablas se puede observar el desglose de cada uno de ellos.

\begin{table}[H]
	\centering
	\caption{Coste del hardware.}
	\begin{tabular}{lccc}
		\toprule
		\textbf{Hardware} & \textbf{Coste (\euro)} & \textbf{Unidades} & \textbf{Total (\euro)} \\
		\midrule
		Ordenador           & \EUR{1,442}            & 1                 & \EUR{1,442}            \\
		Ratón               & \EUR{23.88}            & 1                 & \EUR{23.88}            \\
		Monitor Secundario  & \EUR{40}               & 1                 & \EUR{40}               \\
		Monitor Terciario   & \EUR{130}              & 1                 & \EUR{130}              \\
		Silla ergonómica    & \EUR{325}              & 1                 & \EUR{325}              \\
		Servidor local      & \EUR{200}              & 1                 & \EUR{200}              \\
		\bottomrule
		Total hardware      & -                      & -                 & \EUR{2,160.88}         \\
	\end{tabular}
	\label{tab:coste_hardware}
\end{table}

\begin{table}[H]
	\centering
	\caption{Coste del software.}
	\begin{tabular}{lccc}
		\toprule
		\textbf{Software} & \textbf{Coste (\euro)} & \textbf{Unidades} & \textbf{Total (\euro)} \\
		\midrule
		IDEs JetBrains      & \EUR{289}              & 1                 & \EUR{0}*               \\
		Plugin JPA Buddy    & \EUR{25.99}            & 1                 & \EUR{0}*               \\
		Plugin BashSupport  & \EUR{14}               & 1                 & \EUR{0}*               \\
		Plugin CodeMR       & \EUR{124.24}           & 1                 & \EUR{0}*               \\
		Office 365          & \EUR{100}              & 1                 & \EUR{0}*               \\
		\bottomrule
		Total software      & -                      & -                 & \EUR{0}                \\
	\end{tabular}
	\label{tab:coste_software}
\end{table}

\begin{footnotesize}
	* El coste de estos programas se ha reducido a 0 porque se ha utilizado la licencia de estudiante de la
	Universidad de León para obtenerlos.
\end{footnotesize}
