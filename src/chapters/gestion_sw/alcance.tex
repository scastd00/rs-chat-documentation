\sect{Alcance}{alcance}

\subsect{Definición del proyecto}{definicion_proyecto}

El proyecto consiste en el desarrollo de una aplicación web desde cero para la comunicación e intercambio de archivos
entre los usuarios.\ Inicialmente, se tomó como idea la integración de esta en la plataforma de la universidad.\ Pero
debido a la gran cantidad de requisitos que se tenían que cumplir para poder integrarla en la plataforma, se decidió
optar por una aplicación independiente, conservando la temática de estudiantes, profesores, grados y asignaturas.
% Todo: ¿quitar lo de la integración?
Gracias a los diferentes roles dentro de la aplicación, se puede realizar una comunicación más fluida entre los
usuarios, ya que los profesores pueden crear grupos (además del ya existente para cada asignatura) para, por ejemplo,
dividir a los alumnos en grupos de trabajo y así no se interfiere con el resto del alumnado.\ También cabe la
posibilidad de que cada usuario pueda abrir un chat privado con cualquier otro usuario que esté disponible en los
chats a los que pertenece, abriendo una nueva conversación.\ Los adminstradores son los encargados de crear
las asignaturas y los profesores, así como de gestionar los usuarios de la aplicación.
