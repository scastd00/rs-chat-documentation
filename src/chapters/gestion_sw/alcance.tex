\sect{Alcance}{alcance}

\subsect{Definición del proyecto}{definicion_proyecto}

El proyecto consiste en el desarrollo de una aplicación web para la comunicación mediante mensajes e intercambio de
archivos entre los usuarios.\ Gracias a los diferentes roles en los que se distinguen los usuarios dentro de la
aplicación, se puede
realizar una comunicación más fluida entre los mismos, ya que los profesores pueden crear grupos (además del ya
existente para cada asignatura) para, por ejemplo, dividir a los alumnos en grupos de trabajo y así no interferir con
el resto del alumnado.\ También cabe la posibilidad de que cada usuario pueda abrir un chat privado con cualquier
otro que esté presente en los chats a los que pertenece, abriendo una nueva conversación.\ Los administradores son
los encargados de crear las asignaturas y los profesores, así como de gestionar los usuarios de la aplicación.

Con respecto al diseño, se ha optado por una interfaz sencilla y fácil de usar, minimalista y funcional.\ Se dispone
de adaptabilidad a diferentes dispositivos, por lo que la aplicación es accesible desde cualquier sistema con
conexión a Internet.\ Además, se ha optado por un diseño \textit{responsive}, lo que permite que la aplicación se
adapte a la
pantalla del dispositivo en el que se esté visualizando, ya sea un ordenador de sobremesa, un portátil, una tablet o un
teléfono móvil.

La implementación del proyecto se divide en frontend y backend.\ El frontend es el encargado de mostrar las vistas
y de gestionar las interacciones con el usuario, mientras que el backend es el encargado de controlar la lógica de
la aplicación, la base de datos y la comunicación con el frontend.

Debido a que la aplicación necesita que los usuarios se registren para poder utilizarla, se implementa una política de
privacidad para que los usuarios conozcan cómo se van a tratar sus datos.\ Éstos no se compartirán con
ninguna otra entidad, salvo que el usuario lo autorice.\ Además, se ha introducido un sistema de
recuperación de contraseña para que los usuarios puedan restablecer su contraseña en caso de que la hayan olvidado.

\subsect{Limitaciones de uso}{limitaciones_uso}

Para garantizar el correcto funcionamiento de la aplicación, se establecen las siguientes restricciones de uso:

\begin{itemize}
	\item Se garantiza una buena visualización de la aplicación en dispositivos de cualquier resolución, tanto
	dispositivos móviles como de escritorio, siempre que dispongan de conexión a Internet y un navegador web
	actualizado.
	\item El funcionamiento del sistema se ha probado para el navegador Firefox y los basados en \textbf{Chromium},
	como Google Chrome y Brave.\ No se garantiza el correcto funcionamiento en otros navegadores web.
	\item No se requiere instalación de ningún programa externo.\ De esta manera, se
	simplifica el acceso a la aplicación y se evitan problemas de compatibilidad con el sistema operativo.
	\item Para el almacenamiento de archivos, se utilizará un sistema de almacenamiento en la nube, siguiendo
	una estructura de directorios dependiendo del tipo de archivo que los usuarios suban.
	\item Durante el desarrollo del proyecto, no se utilizarán recursos que incumplan la ley de protección de
	datos ni la ley de propiedad intelectual.
\end{itemize}
\label{itm:alcance_limitaciones}

\subsect{Entregables}{entregables}
% Todo: añadir más cosas

\subsect{Criterios de aceptación}{criterios_aceptacion}

Para que el proyecto se considere aceptado, se deben cumplir una serie de requisitos, que son los siguientes:

\begin{itemize}
	\item Los puntos mencionados en el apartado~\ref{itm:alcance_objetivos} deben estar implementados.
	\item Con el fin de garantizar la ausencia de errores, se debe realizar una serie de pruebas por parte del equipo
	de desarrollo y también por parte de los usuarios.\ En caso de existir algún error, se debe corregir antes de
	que el proyecto sea aceptado.
	\item Los casos de uso deben estar implementados y funcionando correctamente. % Todo: esto es necesario?
	\item Los test unitarios, de aceptación e integración deben pasar sin errores.
\end{itemize}
\label{itm:alcance_criterios_aceptacion}
