\sect{Preparación del servidor}{server_preparation}
Cuando se ha recibido el ordenador, se han seguido una serie de pasos para configurarlo de la manera más segura posible,
ya que se va a utilizar para alojar el backend una aplicación web.\ Estos pasos se describen a continuación.

\subsect{Instalación del sistema operativo}{os_installation}
El sistema operativo que se ha instalado en el ordenador es \boldFont{Ubuntu Server 20.04 LTS}.\ Se ha escogido esta
distribución de Linux debido a que es una de las más populares y con una gran comunidad de usuarios, lo que
hace que sea fácil encontrar información sobre cómo configurar el sistema.\ Además, se ha elegido la versión LTS para
tener soporte durante un periodo de tiempo más largo (en el caso de esta versión, hasta 2025 y con actualizaciones de
seguridad hasta 2030).\ Se han instalado todos los paquetes necesarios para administrarlo de una manera más cómoda
y para garantizar la seguridad e integridad del sistema.\ Algunos de estos paquetes son los siguientes:
\begin{itemize}
	% Todo: lo genera Copilot, poner bibliografía de algo y corregir
	\item \boldFont{OpenSSH}: el protocolo \boldFont{ssh} permite establecer conexiones seguras entre dos
	ordenadores.\ Se ha configurado esta herramienta para que se pueda acceder a él mediante un par de claves
	(pública/privada), eliminando la posibilidad de acceder mediante una contraseña.\ Esto último se realiza para
	evitar que se pueda establecer una conexión con él de forma no autorizada.\ De esta manera, los únicos usuarios
	que pueden acceder al servidor son aquellos que tienen la clave privada asociada a la clave pública registrada en
	el servidor.\ Para generar estas claves, se ha utilizado el comando \boldFont{ssh-keygen} (en el cliente) y se ha
	guardado la clave pública en el archivo \boldFont{authorized\_keys} del servidor.

%	\item \boldFont{ufw}: es un cortafuegos que permite configurar las reglas de entrada y salida de paquetes.
%	\item \boldFont{unattended-upgrades}: es un servicio que permite actualizar automáticamente el sistema operativo
%	cuando se detectan actualizaciones de seguridad.
	\item \boldFont{certbot}: es un servicio que permite obtener certificados SSL y configurarlos automáticamente en el
	servidor web.
\end{itemize}
\label{itm:os_packages}

\subsect{Configuración de la red}{network_configuration}
El ordenador se ha conectado a la red mediante un cable Ethernet para una mejor comunicación entre sistemas.\
En la configuración del router, se ha asignado una dirección IP estática (dentro de la red local) al
ordenador para que siempre tenga la misma dirección IP.\ Se ha contactado con el proveedor de Internet para consultar
si sería posible obtener una dirección IP estática para el ordenador, pero no se ha podido conseguir, ya que
utilizan la tecnología CG-NAT.\ Esto no es un problema ya que existe un servicio llamado \boldFont{DuckDNS} que permite
asociar una dirección IP de un ordenador a un dominio.\ Para configurar este servicio, se ha seguido la guía de
instalación que se encuentra en la página web de \boldFont{DuckDNS}~\cite{DuckDNSi96:online}.
Este servicio requiere de una tarea CRON para que se actualice la dirección IP cada 5 minutos.\ Esto se ha realizado con
un pequeño script en Bash, que envía al servidor de \boldFont{DuckDNS} la dirección IP pública del ordenador.

%\subsect{Configuración del firewall}{firewall_configuration}

\subsect{Instalación y ejecución de la aplicación}{app_installation_and_execution}
La aplicación se ha instalado en el ordenador descargando el repositorio con el código fuente.\ Se ha programado un
script
en Bash que realiza todas las tareas necesarias antes de ejecutar la aplicación en los entornos de producción y
desarrollo.
Este script se ejecuta de manera manual cada vez que se quiere actualizar la aplicación, y realiza las siguientes
tareas:
\begin{itemize}
	\item Actualiza el código fuente de la aplicación.
	\item Instala las dependencias y compila el código fuente.
	\item Exporta las variables de entorno.
	\item Ejecuta la aplicación: para que se ejecute la nueva versión de la aplicación, se manda la señal
	\monoFont{TERM} a todos los procesos que estén en escucha por los puertos que se necesitan para la aplicación
	(\boldFont{4040} - Producción y
	\boldFont{4041} - Desarrollo), configurados en ficheros \monoFont{.env}.\ Esto provoca la terminación los procesos
	que estén escuchando por esos puertos.
\end{itemize}
\label{itm:script_tasks}

\subsect{Configuración de la aplicación}{app_configuration}
Como se ha mencionado anteriormente, la aplicación cuenta con dos entornos: producción y desarrollo.\ Para configurar
cada uno de ellos, se ha creado un fichero \monoFont{.env} en la raíz del proyecto.\ Estos ficheros contienen las
variables de entorno que se necesitan para ejecutar la aplicación.\ De esta manera, se asegura que la aplicación
no exponga ninguna información sensible.\ Las variables más importantes
que se han configurado en estos ficheros son las siguientes:
\begin{itemize}
	\item Usuario y contraseña de la base de datos.
	\item Una cadena de texto que se utiliza para firmar los tokens de autenticación.
	\item Claves de acceso a los servicios de AWS\@.
	\item Puerto en el que se ejecuta la aplicación.
	\item Cuenta y contraseña (de aplicación) para enviar correos electrónicos.
	\item Nombre y contraseña de los ficheros que habilitan SSL en el servidor web.
\end{itemize}
\label{itm:env_variables}

\subsect{Configuración de la base de datos}{db_configuration}
La base de datos se ha configurado por defecto con un usuario nuevo para el uso con la aplicación.\ Este usuario tiene
permisos de lectura y escritura sobre la base de datos, por lo que no es necesario crear un usuario para cada
aplicación.

\subsect{Seguridad del servidor}{server_security}
Con el objetivo de prevenir ataques a los dispositivos de la red local, se han configurado diferentes medidas de
seguridad en el
ordenador y el router.\ Estas medidas son las siguientes:
\begin{itemize}
	\item \boldFont{fail2ban}: es un servicio que permite bloquear las conexiones a un ordenador cuando se detecta un
	ataque de fuerza bruta a la IP de este servidor.
	\item Se han desactivado los puertos inactivos del ordenador, para evitar que se puedan utilizar para
	atacar a otros ordenadores.
	\item En el router, solo se ha habilitado la redirección de puertos a los que se necesitan para ejecutar la
	aplicación.
\end{itemize}

% Todo: configurar Nagios, Zabbix, Snort en el servidor
