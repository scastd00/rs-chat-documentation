\sect{Preparación del servidor}{server_preparation}
Cuando se ha recibido el ordenador, se han seguido una serie de pasos para configurarlo de la manera más segura posible,
ya que se va a utilizar para alojar el backend una aplicación web.\ Estos pasos se describen a continuación.

\subsect{Instalación del sistema operativo}{os_installation}
El sistema operativo que se ha instalado en el ordenador es \boldFont{Ubuntu Server 20.04 LTS}.\ Se ha escogido esta
distribución de Linux debido a que es una de las más populares y que tiene una gran comunidad de usuarios, lo que
hace que sea fácil encontrar información sobre cómo configurar el sistema.\ Además, se ha elegido la versión LTS para
tener soporte durante un periodo de tiempo más largo.\ Se han instalado todos los paquetes necesarios para
establecer conexiones con este ordenador de forma remota, para administrarlo de una manera más cómoda.\ Estos paquetes
son los siguientes:
\begin{itemize}
	% Todo: lo genera Copilot, poner bibliografía de algo y corregir
	\item \boldFont{OpenSSH}: \boldFont{ssh} es un protocolo que permite establecer conexiones seguras entre dos
	ordenadores.\ Se ha
	configurado esta herramienta para que se pueda acceder a él mediante un par de claves (pública/privada), de forma
	que no sea
	necesario introducir una contraseña cada vez que se quiera acceder al ordenador.
%	\item \boldFont{ufw}: es un cortafuegos que permite configurar las reglas de entrada y salida de paquetes.
	\item \boldFont{fail2ban}: es un servicio que permite bloquear las conexiones a un ordenador cuando se detecta un
	ataque de fuerza bruta a la IP de este servidor.
%	\item \boldFont{unattended-upgrades}: es un servicio que permite actualizar automáticamente el sistema operativo
%	cuando se detectan actualizaciones de seguridad.
	\item \boldFont{certbot}: es un servicio que permite obtener certificados SSL y configurarlos automáticamente en el
	servidor web.
\end{itemize}
\label{itm:os_packages}

\subsect{Configuración de la red}{network_configuration}
El ordenador se ha conectado a la red mediante un cable Ethernet.\ En la configuración del router, se ha asignado
una dirección IP estática (dentro de la red local) al
ordenador para que siempre tenga la misma dirección IP.\ Se ha contactado con el proveedor de Internet para consultar
si se puede obtener una dirección IP estática para el ordenador, pero no se ha podido conseguir.\ Esto no es un
problema ya que existe un servicio llamado \boldFont{DuckDNS} que permite asociar
una dirección IP de un ordenador a un dominio.\ Para configurar este servicio, se ha seguido la guía de
instalación que se encuentra en la página web de \boldFont{DuckDNS}. % Todo: insertar enlace a la guía
Este servicio requiere de una tarea CRON para que se actualice la dirección IP cada 5 minutos.\ Esto se ha realizado con
un pequeño script en Bash, que envía al servidor de \boldFont{DuckDNS} la dirección IP pública del ordenador.

\subsect{Configuración del firewall}{firewall_configuration}

\subsect{Instalación de la aplicación}{app_installation}

\subsect{Configuración de la aplicación}{app_configuration}

\subsect{Configuración de la base de datos}{db_configuration}

\subsect{Configuración de la aplicación web}{webapp_configuration}
