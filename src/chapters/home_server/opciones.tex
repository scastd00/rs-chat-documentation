\sect{Problemas en el desarrollo}{development_problems}
Uno de los problemas más significativos que han surgido en el desarrollo de la aplicación se ha dado en la parte del
despliegue en producción de la misma.\ En un comienzo, la aplicación se había desplegado en dos clusters con el plan
gratuito de Heroku (uno para el frontend y otro para el backend), pero en el momento en que esta plataforma publicó
que se dejaría de dar soporte a las aplicaciones gratuitas, se empezaron a buscar alternativas para alojar la
aplicación.

\subsect{Alternativas a Heroku}{heroku_alternatives}
Las opciones que se consideraron fueron las que se describen a continuación, ordenadas según la fecha de consideración.

\begin{itemize}
	\item Para la parte de \boldFont{frontend}, la elección fue muy sencilla. \boldFont{Vercel} fue la primera opción
	a considerar, siendo una plataforma muy conocida por su facilidad para desplegar aplicaciones web.\ Cada vez que se
	realiza un cambio en el código fuente de la aplicación, se inicia un despliegue automático en la plataforma (esto
	ocurre si se tiene enlazado el repositorio de Git).

	\item Para la parte de \boldFont{backend}, se consideraron las siguientes opciones:
	\begin{itemize}
		\item \boldFont{Alojamiento en la nube}: se consideró la posibilidad de alojar la aplicación en la nube utilizando otros
		servicios.\ Se llegó a concretar una reunión con un \textit{Cloud Consultant} de \boldFont{Platform.sh}, que
		fue la primera opción a considerar, pero no se llegó a ningún acuerdo para conseguir un plan gratuito para
		desplegar la aplicación, por tanto, se descartó esta opción.

		\item \boldFont{Raspberry Pi 4}: es un ordenador de tamaño muy reducido, que se puede configurar para que actúe como un
		servidor.\ Es una buena opción en caso de se desee un consumo de energía bajo.\ Sin embargo, no es una opción
		viable para el proyecto a largo plazo, ya que el almacenamiento se realiza en una tarjeta de memoria, cuya
		velocidad de lectura/escritura es más baja en comparación con los discos duros sólidos de los ordenadores
		convencionales.

		\item \boldFont{Servidor propio}: es la opción más viable, ya que se puede aprovechar para realizar otras tareas como
		almacenamiento de gran cantidad datos o desplegar otras aplicaciones más potentes que con la Raspberry.
		Esta ha sido la \boldFont{opción elegida} finalmente, ya que se ha encontrado un pequeño ordenador que cumple con los
		requisitos necesarios para alojar la aplicación.
	\end{itemize}
\end{itemize}

