\subsect{Sprint 9}{sprint9}

\underline{Fecha de inicio}: 11/02/2023

\underline{Fecha de fin}: 11/03/2023

\underline{Objetivos}:
\begin{itemize}
	\item Mejor implementación de las respuestas HTTP\@.
	\item \boldFont{Uso de un modelo de inteligencia artificial para la detección de imágenes inapropiadas.}
\end{itemize}

\underline{Descripción}:
Para este sprint se realizará un completo rediseño de la implementación de las respuestas HTTP, ya que se debe
facilitar su uso y la creación de nuevas respuestas en cada controlador.

Se utilizará un modelo de inteligencia artificial para la detección de imágenes inapropiadas.\ Se realizará un
estudio de los modelos de inteligencia artificial existentes para la detección de imágenes inapropiadas y se elegirá
el más adecuado para el proyecto.\ Para la aplicación se ha usado un modelo pre-entrenado de código abierto que se ha
obtenido del repositorio de GitHub
\href{https://github.com/GantMan/nsfw_model}{\textit{GantMan/nsfw\_model}}~\cite{nsfw-model-repo}.
La implementación de este modelo se realizará en un nuevo servicio utilizando NodeJS y utilizando la librería
TensorFlowJS\@.
