\sect{Procesamiento de los mensajes}{procesamiento-mensajes}

Como hemos visto en la sección anterior, los mensajes que se reciben en el servidor pueden ser de diferentes tipos,
cambiando la forma en que se procesan.\ A continuación se muestra una lista con las acciones que se realizan para
cada tipo de mensaje que se trata:

\begin{itemize}
	\item \boldFont{Text, Image, Video y Audio}: se envían al resto de usuarios del chat en el mismo formato en que llegaron al
	servidor.\ Estos cuatro tipos de mensaje contienen texto exclusivamente, siendo un mensaje o el enlace a un
	archivo almacenado en el bucket de S3.
	\item \boldFont{ActiveUsers}: se envía solo al cliente que lo ha solicitado, y contiene una lista con los usuarios que están
	conectados en ese momento al chat, ordenados alfabéticamente.
	\item \boldFont{GetHistory}: se envía al usuario que lo solicita, y contiene una lista con los últimos 65 mensajes que se han
	enviado al chat.\ Se realiza una lectura del fichero de texto que contiene el historial de mensajes (almacenado
	en disco) y un filtrado de los mensajes de actividad (los 2 siguientes) del solicitante, ya que no son relevantes.
	\item \boldFont{UserJoined y UserLeft}: se notifica a las personas conectadas el nombre del usuario que se ha unido o ha
	salido del chat, % todo (implementar): y se envía a todos los usuarios la lista de usuarios conectados actualizada.
	\item \boldFont{Ping}: se envía un mensaje con un breve texto.\ Se utiliza exclusivamente para mantener la conexión WebSocket
	abierta.
\end{itemize}
