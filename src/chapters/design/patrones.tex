\sect{Patrones de diseño utilizados}{design-patterns}

Definición: \textit{Se trata de una solución que se puede aplicar a diferentes contextos y que se
puede reutilizar en diferentes proyectos.} \\
En este proyecto se han utilizado varios patrones de diseño, para permitir una mejor escalabilidad, mantenibilidad y
reutilización del código.\ A continuación se detallan los patrones utilizados y su justificación.

\subsect{Builder}{builder}


\subsect{Singleton}{singleton}
Este patrón se utiliza para garantizar que una clase tenga una única instancia y proporciona un punto de acceso
global a ella~\cite{sarcar2018java}.\ En el contexto de esta aplicación, se utiliza en ciertas
clases de utilidad y en las clases que asocian rutas a un método HTTP (por ejemplo \mono{/login} con el método POST).
Estas últimas son clases internas de \mono{Routes.java} y los nombres dependen del método HTTP que se debe utilizar para
realizar una petición a una ruta específica.

\begin{table}[ht]
	\centering
	\label{tab:routes}
	\begin{tabular}{|c|c|}
		\hline
		Método HTTP & Clase de la ruta   \\
		\hline
		GET           & \mono{GetRoute}    \\
		POST          & \mono{PostRoute}   \\
		PUT           & \mono{PutRoute}    \\
		DELETE        & \mono{DeleteRoute} \\
		\hline
	\end{tabular}
	\caption{Relación entre método HTTP y ruta.}
\end{table}

Se han utilizado diferentes formas de acceso a la instancia de las clases.\ En el caso de las clases que asocian rutas a
métodos HTTP, el modificador de acceso a la instancia es público.\ En otros casos, se provee un método estático para
obtener la instancia de la clase, como se muestra en el siguiente ejemplo:

\begin{codeBlock}
	\begin{minted}[
		baselinestretch=1.1,
		fontsize=\footnotesize,
		tabsize=4,
	]{java}
public class Routes {
	private Routes() {}
	...
	public static class GetRoute {
		public static final GetRoute INSTANCE = new GetRoute();

		private GetRoute() {}

		public static final String USERS_URL = V_1 + "/users";

		/* Creates an array containing the routes allowed by the low tier user */
		public String[] lowTierRoutes() {...}

		/* Creates an array containing the routes allowed by the medium tier user */
		public String[] mediumTierRoutes() {...}

		/* Creates an array containing the routes allowed by the top tier user */
		public String[] topTierRoutes() {...}
	}
	...
}
	\end{minted}

	\caption{hola}
\end{codeBlock}

\subsect{Strategy}{strategy}
\lipsum
