\sect{Patrones de diseño}{design-patterns}

\textit{Un patrón de diseño es una solución que se aplica para resolver problemas comunes en el
diseño de software}~\cite{designPatternDefinition}.
Actúan como plantillas que se pueden personalizar para resolver un problema concreto.
Hay varias categorías de patrones de diseño, cada una con una finalidad diferente~\cite{designPatternClassification}:
\begin{itemize}
	\item \textbf{Patrones de creación}: se utilizan para crear objetos de una forma flexible y reutilizando código
	existente.
	\item \textbf{Patrones estructurales}: se utilizan para convertir clases y objetos en estructuras más complejas.
	\item \textbf{Patrones de comportamiento}: se utilizan para definir la interacción entre objetos.
\end{itemize}
\label{itm:categorias-patrones}

En este proyecto se han utilizado varios patrones de diseño, para permitir una mejor
\boldFont{escalabilidad}, \boldFont{mantenibilidad} y \boldFont{reutilización} del código.
A continuación, se detalla la información que se incluye de los patrones utilizados:
\begin{itemize}
	\item Definición / categoría.
	\item Explicación de cómo se ha implementado en el proyecto.
	\item Diagrama UML\@.
	\item Justificación de su uso en la aplicación.
\end{itemize}
\label{itm:uso-patrones}

\subsect{Builder}{builder}

Es un patrón de \boldFont{creación} que permite instanciar objetos complejos de una forma sencilla.\ Se ha creado una
interfaz genérica (\monoFont{Builder}) para su reutilización en caso de ser necesaria para cualquier otra clase.
Esta interfaz define el método \monoFont{build()} que devolverá la instancia de un objeto con los datos establecidos
previamente.\ El diagrama UML del patrón implementado es el siguiente:

\begin{umlDiagram}
	\centering
	\begin{tikzpicture}
		\umlinterface[x=-8,y=2]{Builder}{}{
			+ build(): Object
		}{}
		\umlclass[y=-1]{JsonMessageWrapper}{
			- rawPayload: String \\
			- parsedPayload: JsonObject
		}{
			+ getRawPayload(): String \\
			+ getParsedPayload(): JsonObject \\
			+ headers(): JsonObject \\
			+ body(): JsonObject \\
			+ username(): String \\
			+ chatId(): String \\
			+ sessionId(): long \\
			+ type(): String \\
			+ date(): long \\
			+ token(): String \\
			+ content(): String \\
			+ updateDateTime(): void \\
			+ setContent(String): void \\
			+ toString(): String \\
			\umlstatic{+ fromString(String): JsonMessageWrapper} \\
			+ correctStructure(): boolean \\
			+ builder(): BuilderImpl
		}{}
		\umlclass[x=-8,y=-3]{BuilderImpl}{
			- headers: JsonObject \\
			- body: JsonObject
		}{
			+ username(String): BuilderImpl \\
			+ chatId(String): BuilderImpl \\
			+ sessionId(long): BuilderImpl \\
			+ type(String): BuilderImpl \\
			+ date(long): BuilderImpl \\
			+ token(String): BuilderImpl \\
			+ content(String): BuilderImpl
		}{}

		\umlnest{JsonMessageWrapper}{BuilderImpl}
		\umlimpl{BuilderImpl}{Builder}
	\end{tikzpicture}

	\caption{Patrón Builder empleado en la aplicación.}
\end{umlDiagram}

El patrón builder se ha usado en la aplicación para simplificar la creación de objetos de tipo
\monoFont{JsonMessageWrapper}, por el
momento.\ Esta clase es la encargada de encapsular los mensajes que se envían a través de la red en formato JSON\@.


\subsect{Singleton}{singleton}

También es un patrón de \boldFont{creación}.\ Se emplea para garantizar que una clase concreta tenga una única
instancia y proporciona un punto de acceso global a ella~\cite{sarcar2018java}.\ En el contexto de esta aplicación,
se utiliza en ciertas clases de utilidad y en las clases que asocian rutas a un método HTTP (por ejemplo,
\monoFont{/login} con el método POST).\ Estas últimas son clases internas de \monoFont{Routes.java} y los nombres
dependen del método HTTP que se debe utilizar para realizar una petición a una ruta específica.

\begin{table}[H]
	\centering
	\begin{tabular}{|c|c|}
		\hline
		Método HTTP & Clase de la ruta       \\
		\hline
		GET           & \monoFont{GetRoute}    \\
		POST          & \monoFont{PostRoute}   \\
		PUT           & \monoFont{PutRoute}    \\
		DELETE        & \monoFont{DeleteRoute} \\
		\hline
	\end{tabular}
	\caption{Relación entre método HTTP y ruta.}
	\label{tab:routes}
\end{table}

\begin{umlDiagram}
	\centering
	\scalebox{0.8} {
		\begin{tikzpicture}
			\umlclass{Routes}{
				\umlstatic{- V\_1: String = \str{/api/v1}} \\
				\umlstatic{+ ROOT\_URL: String = \str{/}} \\
				\umlstatic{+ ALL\_ROUTES: String = \str{/**}} \\
				\umlstatic{+ ACTUATOR\_URL: String = \str{/actuator/prometheus}} \\
				\umlstatic{+ TEST\_URL: String = \str{/api/v1/test}} \\
				\umlstatic{+ WS\_CHAT\_ENDPOINT: String = \str{/ws/rschat}}
			}{}{}

			\umlclass[x=-4,y=5]{GetRoute}{
				\umlstatic{+ INSTANCE: GetRoute}
			}{
				+ lowTierRoutes(): String[] \\
				+ mediumTierRoutes(): String[] \\
				+ topTierRoutes(): String[]
			}{}
			\umlclass[x=4,y=5]{PostRoute}{
				\umlstatic{+ INSTANCE: PostRoute}
			}{
				+ lowTierRoutes(): String[] \\
				+ mediumTierRoutes(): String[] \\
				+ topTierRoutes(): String[]
			}{}
			\umlclass[x=-4,y=-5]{PutRoute}{
				\umlstatic{+ INSTANCE: PutRoute}
			}{
				+ lowTierRoutes(): String[] \\
				+ mediumTierRoutes(): String[] \\
				+ topTierRoutes(): String[]
			}{}
			\umlclass[x=4,y=-5]{DeleteRoute}{
				\umlstatic{+ INSTANCE: DeleteRoute}
			}{
				+ lowTierRoutes(): String[] \\
				+ mediumTierRoutes(): String[] \\
				+ topTierRoutes(): String[]
			}{}

			\umlnest{Routes}{GetRoute}
			\umlnest{Routes}{PostRoute}
			\umlnest{Routes}{PutRoute}
			\umlnest{Routes}{DeleteRoute}
		\end{tikzpicture}
	}

	\caption{Patrón Singleton empleado en la aplicación.}
\end{umlDiagram}

Se han utilizado diferentes formas de acceso a las instancias de las clases.\ En el caso de las que asocian
rutas a métodos HTTP, el modificador de acceso a la instancia es público.\ En otros casos, se provee un método
estático para obtener la instancia de la clase.

\subsect{Strategy}{strategy}

Este patrón de \boldFont{comportamiento} se utiliza para encapsular un algoritmo dentro de una clase, de forma que
pueda ser intercambiado por otro algoritmo en tiempo de ejecución.\ En la aplicación, se emplea para encapsular el
proceso de manejo de los mensajes que se envían entre los usuarios.\ Existen varias clases que se encargan de
realizarlo, por lo que todas ellas implementan la interfaz \monoFont{MessageStrategy}, que define el método
\monoFont{handle(MessageHandlingDTO)}, encargado de realizar el procesamiento del mensaje.\ Este método recibe un
parámetro de tipo \monoFont{MessageHandlingDTO}, que contiene la información necesaria para realizar el procesamiento
del mensaje:

\begin{itemize}
	\item El mensaje que se debe manejar.
	\item Otros datos que puedan ser necesarios para el manejo del mensaje.\ Dependiendo de la clase, este parámetro
	puede contener más o menos datos.
\end{itemize}
\label{itm:parametershandlestrategy}

La relación entre el tipo de mensaje y las clases encargadas de procesarlo se pueden ver en la
tabla~\ref{tab:messageStrategyRelationship}.
%% Iría aquí la tabla, pero queda un trozo en blanco muy grande si la pongo aquí.

\begin{umlDiagram}
	\centering
	\begin{tikzpicture}
		\umlinterface[x=0,y=0]{MessageStrategy}{}{
			+ handle(MessageHandlingDTO): void
		}{}
	\end{tikzpicture}

	\caption{Interfaz MessageStrategy.}
\end{umlDiagram}

Este patrón se ha utilizado para simplificar el manejo de los mensajes recibidos en el servidor y para que sea más
fácil de extender.\ En el caso de que se desee agregar un nuevo tipo de mensaje, se debe crear una nueva clase que
implemente la interfaz \monoFont{MessageStrategy} y agregarla a la lista de estrategias, que se encuentra en la
clase \monoFont{MessageStrategyMappings}.\ Esta clase es la encargada de determinar qué estrategia se debe utilizar para
manejar el mensaje.\ Para esto, se utiliza el método \monoFont{decideStrategy(receivedMessageType: Message)} que
recibe como parámetro el tipo de mensaje y devuelve la estrategia que se debe utilizar para manejarlo.

\begin{table}[H]
	\centering
	\begin{tabular}{|c|c|}
		\hline
		\textbf{Mensaje}              & \textbf{Clase de la estrategia}       \\ \hline
		\texttt{USER\_CONNECTED}        & \monoFont{UserConnectedStrategy}      \\ \hline
		\texttt{USER\_DISCONNECTED}     & \monoFont{UserDisconnectedStrategy}   \\ \hline
		\texttt{USER\_TYPING}           & \monoFont{UserTypingStrategy}         \\ \hline
		\texttt{USER\_STOPPED\_TYPING}  & \monoFont{UserStoppedTypingStrategy}  \\ \hline
		\texttt{USER\_JOINED}           & \monoFont{UserJoinedStrategy}         \\ \hline
		\texttt{USER\_LEFT}             & \monoFont{UserLeftStrategy}           \\ \hline
		\texttt{TEXT\_MESSAGE}          & \monoFont{TextMessageStrategy}        \\ \hline
		\texttt{IMAGE\_MESSAGE}         & \monoFont{ImageMessageStrategy}       \\ \hline
		\texttt{AUDIO\_MESSAGE}         & \monoFont{AudioMessageStrategy}       \\ \hline
		\texttt{VIDEO\_MESSAGE}         & \monoFont{VideoMessageStrategy}       \\ \hline
		\texttt{PDF\_MESSAGE}           & \monoFont{PdfMessageStrategy}         \\ \hline
		\texttt{TEXT\_DOC\_MESSAGE}     & \monoFont{TextDocMessageStrategy}     \\ \hline
		\texttt{PARSEABLE\_MESSAGE}     & \monoFont{ParseableMessageStrategy}   \\ \hline
		\texttt{ACTIVE\_USERS\_MESSAGE} & \monoFont{ActiveUsersStrategy}        \\ \hline
		\texttt{GET\_HISTORY\_MESSAGE}  & \monoFont{GetHistoryStrategy}         \\ \hline
		\texttt{INFO\_MESSAGE}          & \monoFont{InfoMessageStrategy}        \\ \hline
		\texttt{PING\_MESSAGE}          & \monoFont{PingStrategy}               \\ \hline
		\texttt{ERROR\_MESSAGE}         & \monoFont{ErrorMessageStrategy}       \\ \hline
		\texttt{RESTART\_MESSAGE}       & \monoFont{RestartMessageStrategy}     \\ \hline
		\texttt{MAINTENANCE\_MESSAGE}   & \monoFont{MaintenanceMessageStrategy} \\ \hline
	\end{tabular}
	\caption{Relación Mensaje - Estrategia}
	\label{tab:messageStrategyRelationship}
\end{table}

\subsect{Observer}{observer}

Este patrón de \boldFont{comportamiento} permite definir un mecanismo de subscripción para notificar a múltiples
objetos sobre cualquier evento que le ocurra al objeto que están observando~\cite{Observer19:online}.

En la aplicación, se utiliza para incrementar el contador de mensajes, de cada tipo, que el usuario ha enviado.
Para esto, se ha creado una clase, \monoFont{MessageEvent}, de la cual heredan las clases que representan los
diferentes tipos de mensajes que se pueden enviar.

Para la implementación del objeto ``escuchador'', se sobreescriben todos los métodos necesarios de la interfaz
\monoFont{ApplicationListener}.\ El método \monoFont{onApplicationEvent (event: ApplicationEvent)} es el encargado de
manejar el evento lanzado, el cual se recibe como parámetro.

\begin{umlDiagram}
	\centering
	\begin{tikzpicture}
		\umlclass[x=0,y=0]{MessageEvent}{
			- type: String \\
			- username: String \\
			\# callback: Function<String, Void>
		}{
			+ MessageEvent(source: Object, type: String, username: String) \\
			+ getType(): String \\
			+ setType(type: String): void \\
			+ getUsername(): String \\
			+ setUsername(username: String): void \\
			+ getCallback(): Function<String, Void> \\
			+ setCallback(callback: Function<String, Void>): void
		}{}
	\end{tikzpicture}

	\caption{Clase que representa los eventos de envío de mensajes.}
\end{umlDiagram}

\begin{umlDiagram}
	\centering
	\begin{adjustbox}{width=\textwidth}
		\begin{tikzpicture}
			\umlclass[x=0,y=-4]{MessageEventListener}{
				- userBadgeService: UserBadgeService \\
				- userService: UserService
			}{
				+ MessageEventListener(userBadgeService: UserBadgeService, userService: UserService) \\
				+ onApplicationEvent(event: MessageEvent): void
			}{}

			\umlinterface[x=0,y=0]{ApplicationListener}{}{
				+ onApplicationEvent(event: E): void
			}{}

			\umlimpl{MessageEventListener}{ApplicationListener}
		\end{tikzpicture}
	\end{adjustbox}

	\caption{Clase que representa el ``escuchador'' de los eventos de envío de mensajes.}
\end{umlDiagram}

El uso de este patrón permite que el código sea más mantenible y extensible, ya que se puede agregar un nuevo tipo de
mensaje sin tener que modificar el código existente.\ Solamente habría que crear una nueva clase que herede de
\monoFont{MessageEvent} y sobreescribir los métodos necesarios.
