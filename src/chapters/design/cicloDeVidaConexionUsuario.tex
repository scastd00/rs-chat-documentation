\subsect{Ciclo de vida de la conexión de usuarios}{ciclo-de-vida-conexión-usuario}

Cuando un usuario accede a la aplicación, se inicia una conexión entre el cliente y el servidor a través
del protocolo de comunicación bidireccional \boldFont{WebSocket}~\cite{RFCWebSocket}.
Esta conexión se mantiene abierta mientras el usuario tenga la sesión iniciada en la aplicación, y se cierra cuando
el usuario termina la sesión.
La secuencia de eventos que ocurren durante la conexión de un usuario al chat es la siguiente:

\subsubsect{Frontend}{frontend}

\begin{enumerate}
	\item Se realiza una solicitud de conexión WebSocket al servidor al iniciar sesión.
	\item Se recibe la respuesta del servidor, que se acepta si el código de estado HTTP es 101 (código de estado
	\monoFont{SWITCHING\_PROTOCOLS}) y se crea un temporizador para enviar mensajes de tipo \monoFont{PING\_MESSAGE}
	cada 30 segundos, manteniendo la conexión activa hasta que el usuario cierre sesión.\ En caso contrario, no se
	establece la conexión.
	\item Cuando el usuario desea entrar a un chat, se realiza una petición HTTP para comprobar que puede acceder.
	Esto se realiza para
	que, en caso de que el usuario introduzca de forma manual la URL de un chat al que no tiene acceso, se le redirija
	a la página de inicio de la aplicación.
	\item Si se confirma que el usuario \boldFont{puede acceder} al chat:
	\begin{enumerate}
		\item Se envía un mensaje de tipo \monoFont{USER\_JOINED} al servidor.
		\item Se consultan los últimos mensajes del historial del chat con el mensaje de tipo
		\monoFont{GET\_HISTORY\_MESSAGE}.
		\item Se realiza una petición de la lista de usuarios activos con un mensaje de tipo
		\monoFont{ACTIVE\_USERS\_MESSAGE}.
	\end{enumerate}
	\item Si el usuario \boldFont{no puede acceder} al chat, se le redirige a la página principal.
\end{enumerate}
\label{itm:frontend-connection-life-cycle}

\subsubsect{Backend}{backend}

Cuando comienza la ejecución del programa, se indica a Spring Boot que el manejador de mensajes a través de WebSocket
del servidor es una instancia de la clase \monoFont{WebSocketHandler}, en la ruta
\quoted{/ws/rschat}.\ En esta clase se redefine el método \monoFont{handleTextMessage} de la clase
\monoFont{AbstractWebSocketHandler} de Spring Boot, que es el encargado de gestionar los mensajes de texto recibidos
por el servidor.\ El proceso de \boldFont{conexión} al servidor sigue este flujo de eventos:

\begin{enumerate}
	\item Se establece la conexión WebSocket con el usuario.\ Esto ocurre de forma transparente al programador debido a
	que la implementación se realiza en el framework de Spring Boot.
	\item Se recibe el mensaje \monoFont{USER\_JOINED} del cliente y se crea un objeto de tipo \monoFont{Client}
	(formado por la instancia de \monoFont{WebSocketSession} y el \monoFont{ClientID} del usuario).\ Este nuevo
	objeto se añade a la lista de usuarios conectados al chat.
	\begin{enumerate}
		\item Si el usuario es el primero que se ha conectado al chat, se crea uno nuevo, guardándose en la lista de
		chats.
		\item Si no, se añade al chat correspondiente de la lista de chats.
	\end{enumerate}
	\item Se recibe el mensaje \monoFont{GET\_HISTORY\_MESSAGE} del cliente y se envían como respuesta los últimos 65
	mensajes del historial de mensajes del chat.
	\item Se recibe el mensaje \monoFont{ACTIVE\_USERS\_MESSAGE} del cliente y se devuelve la lista con los usuarios
	activos en el chat.
\end{enumerate}
\label{itm:backend-connection-life-cycle}
Y el proceso de \boldFont{desconexión} se realiza como sigue:

\begin{enumerate}
	\item Se recibe el mensaje \monoFont{USER\_LEFT} del cliente.
	\item Se informa al resto de los usuarios del chat de la desconexión del usuario.
	\item Al eliminar un usuario del chat se pueden dar 2 casos:
	\begin{enumerate}
		\item Si el usuario es el último que se ha desconectado del chat, se elimina el chat de la tabla de dispersión
		\monoFont{chats}.\ Antes de que esto ocurra, el historial de mensajes del chat que se haya registrado desde
		que se inició se envía al almacenamiento en la nube.
		\item Si no, se elimina el usuario de la lista de usuarios del chat.
	\end{enumerate}
	\item Se cierra la conexión WebSocket con el usuario, de forma transparente al programador (al igual que la
	conexión).
\end{enumerate}

\begin{umlDiagram}
	\centering

	\begin{tikzpicture}
		\umlclass{ChatManagement}{
			- chats: Map<String, Chat>
		}{
			- chatExists(String): boolean \\
			+ addClientToChat(Client): void \\
			+ removeClientFromChat(ClientID): void \\
			+ broadcastToSingleChatAndSave(String, String): void \\
			+ broadcastToSingleChatWithoutSaving(String, String): void \\
			+ broadcastToSingleChatExcludeClientAndSave(String, ClientID): void \\
			+ broadcastToSingleChatExcludeClientWithoutSaving(String, ClientID): void \\
			+ totalBroadcast(String): void \\
			+ mentionUser(String, String, String): void \\
			+ getActiveUsernamesOfChat(String): List<String> \\
			+ close(): void \\
			+ setClientAway(ClientID): void \\
			+ setClientActive(ClientID): void \\
			+ sendNotificationTo(String, String): void \\
			- saveAllToS3(): void \\
			- deleteUnwantedUsers(): void
		}{}
		\umlnote[y=-7,width=8.5cm]{ChatManagement}{
			Los dos últimos métodos se ejecutan de manera periódica cada 10 y 3 minutos respectivamente.
		}{}
	\end{tikzpicture}

	\caption{Clase \monoFont{ChatManagement} para la gestión de los chats. (Fuente: Elaboración propia).}
	\label{diagram-ChatManagementClass}
\end{umlDiagram}
