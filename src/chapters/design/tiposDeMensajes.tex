\subsect{Tipos de mensajes}{tipos-mensajes}

Como hemos visto anteriormente, los mensajes que se reciben en el servidor pueden ser de diferentes tipos,
cambiando la forma en que se procesan.\ A continuación, se muestra una lista con las acciones que se realizan para
cada tipo de mensaje que se trata:

\begin{itemize}
	\item \boldFont{UserConnected}: es un mensaje que se envía al servidor cuando un usuario se conecta a la
	aplicación.\ Sirve para establecer la conexión WebSocket.
	\item \boldFont{UserDisconnected}: es un mensaje que se envía al servidor cuando un usuario se desconecta de la
	aplicación.\ Sirve para cerrar la conexión WebSocket.
	\item \boldFont{UserTyping y UserStoppedTyping}: no se han implementado todavía, ya que no se han considerado
	indispensables para la interacción.
	\item \boldFont{UserJoined y UserLeft}: se notifica a las personas conectadas el nombre del usuario que se ha
	unido o ha
	salido del chat, y se añade o elimina de la lista de usuarios conectados.
	\item \boldFont{Text, Image, Audio, Video, PDF, TextDoc}: se envían al resto de usuarios del chat en el mismo
	formato en que llegaron al
	servidor.\ Estos cuatro tipos de mensaje contienen texto exclusivamente, siendo un mensaje o el enlace a un
	archivo almacenado en el bucket de S3.
	\item \boldFont{Parseable}: es un tipo especial, que se utiliza para enviar mensajes que contienen información
	adicional.\ Se utiliza para leer parámetros en los comandos.
	\item \boldFont{ActiveUsers}: se envía solo al cliente que lo ha solicitado, y contiene una lista con los usuarios
	que están conectados en ese momento al chat, ordenados alfabéticamente.
	\item \boldFont{GetHistory}: se envía al usuario que lo solicita, y contiene una lista con los últimos 65 mensajes
	que se han enviado al chat.\ Se realiza una consulta del historial de mensajes (almacenado en caché)
	y un filtrado de los mensajes de actividad del solicitante, ya que no son relevantes.
	Además, si el usuario lo desea, se pueden leer los mensajes anteriores a esos 65 iniciales, utilizando
	la paginación, hasta llegar al primer mensaje enviado al chat.
	\item \boldFont{Info, Restart y Maintenance}: son un tipo de mensajes que los administradores pueden enviar a todos
	los chats a la vez.\ Se utilizan para notificar a los usuarios de algún evento que va a ocurrir en el servidor,
	un reinicio o un mantenimiento.
	\item \boldFont{Ping}: se utiliza exclusivamente para mantener la conexión WebSocket abierta.\ Se envía un texto
	corto para no sobrecargar la red, y se recibe el mismo texto como respuesta.
	\item \boldFont{Error}: se envía cuando se produce algún error en el servidor, y contiene un mensaje con la
	descripción del error.
\end{itemize}
