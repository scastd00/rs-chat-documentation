\subsect{Almacenamiento de ficheros}{aws-storage}

Para poder almacenar los historiales de los chats y los ficheros multimedia requeridos por la aplicación, se ha
optado por utilizar el servicio de almacenamiento de ficheros S3 de Amazon Web Services (AWS).
Este servicio permite almacenar ficheros de forma segura y escalable, con una alta disponibilidad
y durabilidad~\cite{AWSAlmac67:online}.\ Además, permite configurar el acceso a los ficheros mediante políticas
y dispone de encriptación del contenido almacenado.

Este servicio se basa en la utilización de \boldFont{buckets}, que son contenedores de ficheros.
Se identifican mediante un nombre único, y se pueden crear tantos como se necesiten.\ Los ficheros que se almacenan
en los buckets se denominan \boldFont{objetos}, y se identifican por una clave única.

Para poder utilizar este servicio de manera programática, es necesario crear un usuario en AWS con permisos de lectura
y escritura en el servicio S3 y obtener las credenciales de acceso.\ Estas credenciales se utilizan para configurar
el cliente de AWS, que permite realizar operaciones sobre los buckets y los objetos.\ En el caso de la aplicación,
se utiliza el cliente de AWS para subir y descargar ficheros multimedia y para almacenar los historiales de los
chats en un bucket de S3.

El bucket utilizado para el almacenamiento se denomina \textit{rs-chat-bucket} y se ha configurado para que el acceso
a todos los recursos que contiene sea público (ver código~\ref{public-s3-bucket-access-policy}).
De esta forma, cualquier usuario puede acceder a los ficheros y también
visualizarlos desde el propio chat sin necesidad de autenticarse en AWS.\ Para ello, se utiliza la URL del fichero
que se obtiene al subirlo al bucket, que se almacena en la base de datos de la aplicación para cada archivo.

\begin{codeBlock}
	\begin{minted}[
		baselinestretch=1.1,
		fontsize=\footnotesize,
		tabsize=4,
		xleftmargin=2.3cm,
	]{json}
{
    "Version": "2012-10-17",
    "Statement": [
        {
            "Sid": "PublicAccessStatement",
            "Effect": "Allow",
            "Principal": "*",
            "Action": "s3:GetObject",
            "Resource": [
                "arn:aws:s3:::rs-chat-bucket/image/*",
                "arn:aws:s3:::rs-chat-bucket/video/*",
                "arn:aws:s3:::rs-chat-bucket/audio/*",
                "arn:aws:s3:::rs-chat-bucket/template/*",
                "arn:aws:s3:::rs-chat-bucket/badges/*",
                "arn:aws:s3:::rs-chat-bucket/application/*"
            ]
        }
    ]
}
	\end{minted}
	\caption{Política de acceso público al bucket de S3.}
	\label{public-s3-bucket-access-policy}
\end{codeBlock}

Para que no existan problemas en el acceso a los ficheros desde la aplicación, se ha configurado la política de CORS
del bucket para que permita el acceso a los recursos desde cualquier origen.

\begin{codeBlock}
	\begin{minted}[
		baselinestretch=1.1,
		fontsize=\footnotesize,
		tabsize=4,
		xleftmargin=4.5cm,
	]{json}
[
	{
		"AllowedHeaders": [
			"*"
		],
		"AllowedMethods": [
			"GET",
			"POST",
			"PUT",
			"DELETE",
			"HEAD"
		],
		"AllowedOrigins": [
			"*"
		],
		"ExposeHeaders": []
	}
]
	\end{minted}
	\caption{Política de CORS del bucket de S3.}
\end{codeBlock}

La interacción con el bucket de S3 desde la aplicación se ha realizado desde una clase denominada \monoFont{S3}, que
implementa métodos para subir, descargar y borrar ficheros del bucket.\ Esta clase utiliza el cliente de AWS
para realizar las operaciones sobre el bucket.\ En el código~\ref{upload-file-to-s3} se muestra el método utilizado
para subir un fichero.

\begin{codeBlock}
	\begin{minted}[
		baselinestretch=1.1,
		fontsize=\footnotesize,
		tabsize=2,
	]{java}
public URI uploadFile(String mediaType, String fileName, byte[] dataBytes,
											JsonObject metadata) {
	String s3Key = this.s3Key(mediaType, fileName);

	Map<String, String> metadataMap = new HashMap<>();
	metadata.asMap().forEach((key, value) -> metadataMap.put(key, value.toString()));

	this.s3Client.putObject(
			PutObjectRequest.builder()
			                .bucket(S3_BUCKET_NAME)
			                .key(s3Key)
			                .metadata(metadataMap)
			                .build(),
			RequestBody.fromBytes(dataBytes)
	);

	return uploadedFileURI(s3Key);
}
	\end{minted}
	\caption{Método para subir un fichero al bucket de S3.}
	\label{upload-file-to-s3}
\end{codeBlock}

Para ello, se utiliza el método \monoFont{putObject} del cliente de AWS S3, que recibe dos parámetros con los tipos y
orden siguientes:

\begin{itemize}
	\item \monoFont{PutObjectRequest}: contiene la información del bucket y del objeto que se va a subir.
	\item \monoFont{RequestBody}: contiene los datos del fichero que se va a subir.\ En este caso se obtiene
	a partir de los datos en binario del fichero.
\end{itemize}
