% Loki docker plugin causes all the containers to be unable to restart or kill. The alternative is to use the
% json-file driver with custom options. The json-file driver is the default driver for Docker.
% https://stackoverflow.com/questions/38567355/docker-compose-global-level-logging
% https://howchoo.com/devops/how-to-add-a-health-check-to-your-docker-container

\sect{Docker}{docker}
Se ha utilizado Docker para la creación de contenedores que ejecutan los servicios de la aplicación.\ Para ello, se ha
utilizado la herramienta \monoFont{docker-compose}, que permite definir y ejecutar aplicaciones Docker de forma
sencilla.\ Para la configuración de los contenedores, se ha utilizado el fichero \monoFont{docker-compose.yml} que se
encuentra en la raíz del proyecto.\ Este fichero define los servicios que se ejecutarán en los contenedores, así como
las dependencias entre ellos.

Debido a que cada vez que se elimina un contenedor (al reiniciar el servidor o por necesidad de crear de nuevo los
contenedores) se pierden los datos, se necesita persistencia de los mismos
y de sus configuraciones.\ Para lograr esto, se han empleado \boldFont{volúmenes}, que son
directorios que se crean en el sistema de archivos del host, y que se montan en los contenedores.\ De esta forma, los
datos no se pierden al parar ni eliminar el contenedor.\ Los volúmenes se definen en el mismo fichero
\monoFont{docker-compose.yml} mediante la opción \monoFont{volumes} y son los siguientes:

\begin{itemize}
	\item \monoFont{rschat-db}: Directorio que contiene los datos y script inicial de la base de datos.
	\item \monoFont{rschat-logs}: Directorio que contiene los ficheros de log de la aplicación.
	\item \monoFont{grafana-storage}: Directorio que contiene la configuración de Grafana.
\end{itemize}

Además, se ha definido un volumen para la base de datos, que se encuentra en el
directorio \monoFont{db}.\ Este volumen se usa para almacenar los datos de la base de datos, de manera que se
puedan recuperar en caso de que se elimine el contenedor.

Todos los contenedores utilizados para la aplicación se mencionan a continuación, junto con una breve descripción de
cada uno de ellos (los marcados con el símbolo \textsuperscript{\textasteriskcentered} son los que se describirán en
detalle más adelante, debido a que tienen una configuración y funcionalidad más compleja a la de los dos primeros):

\begin{itemize}
	\item \monoFont{rschat}: Contenedor que ejecuta el backend de la aplicación.
	\item \monoFont{rschat-db}: Contenedor que ejecuta la base de datos MySQL\@.
	\item \monoFont{prometheus\textsuperscript{\textasteriskcentered}}: Contenedor que ejecuta el servidor de métricas
	Prometheus.
	\item \monoFont{grafana\textsuperscript{\textasteriskcentered}}: Contenedor que ejecuta el panel de observabilidad
	Grafana.
	\item \monoFont{loki\textsuperscript{\textasteriskcentered}}: Contenedor que ejecuta el servidor de logs Loki.
	\item \monoFont{promtail\textsuperscript{\textasteriskcentered}}: Contenedor que ejecuta el agente de logs
	Promtail.
\end{itemize}

% Todo: revisar estas subsecciones para añadir más detalles e imágenes.
\subsect{Prometheus - Agente de métricas}{prometheus}
Prometheus es un servidor de métricas que permite almacenar y consultar métricas de los servicios de la aplicación.\ Se
ha utilizado para almacenar las métricas de la aplicación, así como las métricas de los contenedores de Docker.\ Para
la configuración de Prometheus, se ha utilizado el fichero \monoFont{prometheus.yml} que se encuentra en la raíz del
proyecto.\ Este fichero define los \textit{targets} que se van a monitorizar, así como la configuración de los
\textit{scrapers}.\ Los \textit{targets} que se monitorizan son los siguientes:

\begin{itemize}
	\item \monoFont{rschat}: Se monitoriza el servicio de la aplicación.
	\item \monoFont{rschat-db}: Se monitoriza la base de datos.
	\item \monoFont{node-exporter}: Se monitoriza el servidor.
\end{itemize}

\subsect{Grafana - Panel de observabilidad}{grafana}
Grafana es un panel de observabilidad que permite visualizar las métricas de los servicios de la aplicación.\ Se ha
utilizado para visualizar las métricas de la aplicación, así como las métricas de los contenedores de Docker.\ Se
ha integrado junto con Loki para mostrar los logs de la aplicación.

\subsect{Loki - Servidor de logs}{loki}
Loki es un servidor de logs que permite almacenar y consultar los logs de los servicios de la aplicación.\ Se ha
utilizado para almacenar los logs de la aplicación.\ Para la configuración de Loki, se ha utilizado el fichero
\monoFont{loki.yml} que se encuentra en la raíz del proyecto.\ Este fichero define los \textit{targets} que se van a
monitorizar, así como la configuración de los \textit{scrapers}.\ Los \textit{targets} que se monitorizan son los
siguientes:

\begin{itemize}
	\item \monoFont{rschat}: Se monitorizan los logs de la aplicación.
	\item \monoFont{rschat-db}: Se monitorizan los logs de la base de datos.
	\item \monoFont{node-exporter}: Se monitorizan los logs del servidor.
\end{itemize}

\subsect{Promtail - Agente de logs}{promtail}
Promtail es un agente de logs que permite enviar los logs de los servicios de la aplicación a Loki.\ Se ha utilizado
para enviar los logs de la aplicación.\ Para la configuración de Promtail, se ha utilizado el fichero
\monoFont{promtail.yml} que se encuentra en la raíz del proyecto.\ Este fichero define los \textit{targets} que se van
a monitorizar, así como la configuración de los \textit{scrapers}.\ Los \textit{targets} que se monitorizan son los
siguientes:

\begin{itemize}
	\item \monoFont{rschat}: Se monitorizan los logs de la aplicación.
	\item \monoFont{rschat-db}: Se monitorizan los logs de la base de datos.
	\item \monoFont{node-exporter}: Se monitorizan los logs del servidor.
\end{itemize}
