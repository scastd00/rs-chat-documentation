% Loki docker plugin causes all the containers to be unable to restart or kill. The alternative is to use the
% json-file driver with custom options. The json-file driver is the default driver for Docker.
% https://stackoverflow.com/questions/38567355/docker-compose-global-level-logging
% https://howchoo.com/devops/how-to-add-a-health-check-to-your-docker-container

\sect{Docker}{docker}
Se ha utilizado Docker para la creación de contenedores que ejecutan los servicios de la aplicación.\ Para ello, se ha
utilizado la herramienta \monoFont{docker-compose}, que permite definir y ejecutar aplicaciones Docker de forma
sencilla.\ Para la configuración de los contenedores, se ha utilizado el fichero \monoFont{docker-compose.yml} que se
encuentra en la raíz del proyecto.\ Este fichero define los servicios que se ejecutarán en los contenedores, así como
las dependencias entre ellos.

Para la persistencia de los datos y configuraciones, se han empleado \boldFont{volúmenes}.\ Los volúmenes son
directorios que se crean en el sistema de archivos del host, y que se montan en los contenedores.\ De esta forma, los
datos no se pierden al parar ni eliminar el contenedor.\ Los volúmenes se definen en el fichero
\monoFont{docker-compose.yml} mediante la opción \monoFont{volumes} y son los siguientes:

\begin{itemize}
	\item \monoFont{rschat-db}: Directorio que contiene los datos de la base de datos.
	\item \monoFont{rschat-logs}: Directorio que contiene los ficheros de log de la aplicación.
	\item \monoFont{grafana-storage}: Directorio que contiene la configuración de Grafana.
\end{itemize}

Además, se ha definido un volumen para la base de datos, que se encuentra en el
directorio \monoFont{db}.\ Este volumen se usa para almacenar los datos de la base de datos, de manera que se
puedan recuperar en caso de que se elimine el contenedor.

