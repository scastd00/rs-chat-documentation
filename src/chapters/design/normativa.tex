El proyecto se ha desarrollado siguiendo las directrices de la normativa aplicable en materia de protección de datos
personales, en concreto, el Reglamento (UE) 2016/679 del Parlamento
Europeo y del Consejo, de 27 de abril de 2016, relativo a la protección de las personas físicas en lo que respecta al
tratamiento de datos personales y a la libre circulación de estos datos (RGPD)~\cite{REGLAMEN74:online} y la
Ley Orgánica 3/2018, de 5 de diciembre, de Protección
de Datos Personales y garantía de los derechos digitales (LOPDGDD)~\cite{BOEA20187-LOPDGDD}.

En concreto, se ha tenido en cuenta el artículo 5 del RGPD, que establece los principios relativos al tratamiento de
datos personales; y el artículo 6 del RGPD, que establece las bases legales para el tratamiento de datos personales.
En este caso, el tratamiento de datos personales se basa en el consentimiento del interesado, que se solicita en el
momento del registro en la aplicación.

Además, se ha tenido en cuenta el artículo 32 del RGPD, que establece la
seguridad del tratamiento de datos personales, y el artículo 35 del RGPD, que establece la evaluación de impacto en la
protección de datos.

\sect{Privacidad de los usuarios}{privacidad}

Para garantizar la seguridad de los datos de los usuarios, se ha redactado una política de privacidad
que se puede consultar de manera extensa en siguiente enlace:
\href{https://rschat.vercel.app/privacy}{https://rschat.vercel.app/privacy}.

De manera resumida, esta política garantiza que los datos de los usuarios se almacenan en la base de datos alojada en el
servidor local y que no se comparten con terceros.\ Los únicos datos que se almacenan son los necesarios para el
funcionamiento de la aplicación, como son el nombre de usuario, la contraseña (encriptada), el correo electrónico y
el nombre completo (pero este último solo se muestra en el perfil del propio usuario y no es visible a los demás).
Además, se almacena la fecha de la última conexión del usuario, para determinar si la sesión actual ha caducado o no.

Todos los usuarios tienen el derecho a solicitar la eliminación de sus datos de la base de datos así como
la eliminación de su cuenta de usuario.\ Para ello, se deberá enviar un correo electrónico a la dirección
\href{mailto:rschat.info@gmail.com}{rschat.info@gmail.com} con el asunto ``Solicitud de eliminación de datos'' y se
procederá a eliminar todos los datos del usuario que se soliciten.
