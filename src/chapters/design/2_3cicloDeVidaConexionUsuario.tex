\sect{Ciclo de vida de la conexión de usuarios}{ciclo-de-vida-conexión-usuario}

Cuando un usuario accede a un chat de la aplicación, se inicia una conexión entre el cliente y el servidor a través
del protocolo de comunicación bidireccional \boldFont{WebSocket}~\cite{RFCWebSocket}.\ Esta conexión se mantiene
abierta mientras el
usuario esté en el chat, y se cierra cuando el usuario abandona el chat.\ La secuencia de eventos que ocurren durante
la conexión de un usuario al chat es la siguiente

\subsect{Frontend}{frontend}

\begin{enumerate}
	\item Se realiza una solicitud de conexión WebSocket al servidor.
	\item Se realiza una petición HTTP para comprobar que el usuario puede acceder al chat.\ Esto se realiza para
	que, en caso de que el usuario introduzca la URL de un chat al que no tiene acceso, se le redirija a la página
	de inicio de la aplicación.
	\item Si se confirma que el usuario \boldFont{puede acceder} al chat:
	\begin{enumerate}
		\item Se establece la conexión WebSocket.
		\item Se envía un mensaje de tipo \monoFont{USER\_JOINED} al servidor.
		\item Se consultan los últimos 65 mensajes del historial de mensajes del chat con el mensaje de tipo
		\monoFont{GET\_HISTORY\_MESSAGE}.
		\item Se realiza una petición de la lista de usuarios activos con un mensaje de tipo
		\monoFont{ACTIVE\_USERS\_MESSAGE}.
		\item Se configura un temporizador para mandar un mensaje de tipo \monoFont{PING\_MESSAGE} cada 30 segundos.
		Esto se realiza para que el servidor no cierre la conexión por inactividad.
	\end{enumerate}
	\item Si el usuario \boldFont{no puede acceder} al chat:
	\begin{enumerate}
		\item Se cierra la conexión WebSocket.
		\item Se redirige al usuario a la página principal.
	\end{enumerate}
\end{enumerate}
\label{itm:frontend-connection-life-cycle}

\subsect{Backend}{backend}

Cuando comienza la ejecución del programa, se indica a Spring Boot que el manejador de mensajes a través de WebSocket
del servidor es una instancia de la clase \monoFont{WebSocketHandler}, en la ruta
\quoted{/ws/rschat}.\ Al instanciar esta clase, se crea un objeto \boldFont{chatMap} de tipo
\monoFont{WebSocketChatMap}, que contiene el atributo
\monoFont{chats} (ver Diagrama UML~\ref{diagram-WebSocketChatMapClass}).\ Es una tabla de dispersión que contiene los
chats activos en la aplicación.\ Cada entrada asocia a una cadena de texto (identificador del chat) la instancia de
un objeto de tipo \monoFont{Chat}.\ El proceso de \boldFont{conexión} al servidor sigue el siguiente flujo de eventos:

\begin{enumerate}
	\item Se establece la conexión WebSocket con el usuario.\ Esto ocurre de forma transparente al programador debido a
	que la implementación se realiza en el framework de Spring Boot.
	\item Se recibe el mensaje \monoFont{USER\_JOINED} del cliente y se crea un objeto de tipo \monoFont{WSClient}
	(formado por la instancia de \monoFont{WebSocketSession} y el \monoFont{WSClientID} del usuario).\ Este nuevo
	objeto se añade a la lista de usuarios conectados al chat.
	\begin{enumerate}
		\item Si el usuario es el primero que se ha conectado al chat, se crea uno nuevo, guardándose en la lista de
		chats.
		\item Si no, se añade al chat correspondiente de la lista de chats.
	\end{enumerate}
	\item Se recibe el mensaje \monoFont{GET\_HISTORY\_MESSAGE} del cliente y se envían como respuesta los últimos 65
	mensajes del historial de mensajes del chat.
	\item Se recibe el mensaje \monoFont{ACTIVE\_USERS\_MESSAGE} del cliente y se devuelve la lista con los usuarios
	activos en el chat.
\end{enumerate}
\label{itm:backend-connection-life-cycle}
Y el proceso de \boldFont{desconexión} se realiza como sigue:

\begin{enumerate}
	\item Se recibe el mensaje \monoFont{USER\_LEFT} del cliente.
	\item Se informa al resto de los usuarios del chat de la desconexión del usuario.
	\item Al eliminar un usuario del chat se pueden dar 2 casos:
	\begin{enumerate}
		\item Si el usuario es el último que se ha desconectado del chat, se elimina el chat de la tabla de dispersión
		\monoFont{chats}.\ Cuando esto ocurre, el historial de mensajes del chat que se haya registrado desde que se
		inició, se envía al almacenamiento en la nube.
		\item Si no, se elimina el usuario de la lista de usuarios del chat.
	\end{enumerate}
	\item Se cierra la conexión WebSocket con el usuario, de forma transparente al programador (al igual que la
	conexión).
\end{enumerate}

\begin{umlDiagram}
	\centering

	\begin{tikzpicture}
		\umlclass{WebSocketChatMap}{
			-- Map~<String, Chat>~ chats
		}{
			-- createChat(String): void \\
			-- chatExists(String): boolean \\
			-- getClientsOf(String): List~<WSClient> \\
			-- saveMessage(String, String): void \\
			+ getClient(WSClientID): WSClient \\
			+ addClientToChat(WSClient): void \\
			+ removeClientFromChat(WSClientID): void \\
			+ broadcastToSingleChat(String, String): void \\
			+ broadcastToSingleChatAndExcludeClient(WSClientID, String): void \\
			+ totalBroadcast(String): void \\
			+ getUsernamesOfChat(String): List~<String> \\
			-- saveAllToS3(): void \\
			-- deleteNullUsers(): void
		}{}
		\umlnote[y=-5.5,width=8.5cm]{WebSocketChatMap}{
			Los dos últimos métodos se ejecutan de manera periódica cada 10 y 3 minutos respectivamente.
		}{}
	\end{tikzpicture}

	\caption{Clase \monoFont{Chat} para almacenar los usuarios activos}
	\label{diagram-WebSocketChatMapClass}
\end{umlDiagram}
