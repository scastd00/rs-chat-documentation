\subsect{Seguridad}{seguridad}

Para la prevención de ciberataques al servidor por parte de usuarios malintencionados, se han
establecido las medidas de seguridad que se detallan a continuación:

\begin{itemize}
	\item \textbf{Cifrado de la comunicación entre el servidor y los clientes}: Para evitar que los datos de los
	usuarios puedan ser interceptados por terceros, se ha establecido un cifrado de las comunicaciones.\ Para ello, se
	ha empleado el protocolo TLS, que permite cifrar la comunicación utilizando el certificado SSL que proporciona
	Let's Encrypt.\ Este certificado se ha generado automáticamente mediante su servicio gratuito de certificados y,
	utilizando la aplicación \texttt{Certbot}, se renueva automáticamente cada 90 días.\ El certificado se
	almacena en el servidor en el directorio \texttt{/etc/letsencrypt/live/}.

	\item \textbf{Configuración del servidor ssh}: Para evitar que los usuarios puedan acceder al servidor mediante
	ssh, se ha configurado para que únicamente se pueda acceder mediante claves ssh (no permitiendo usuario y
	contraseña).\ Estas claves se obtienen mediante el comando \texttt{ssh-keygen}, que genera un par de claves
	(pública y privada), y se ha copiado la clave pública del cliente en el fichero
	\texttt{/home/user/.ssh/authorized\_keys} del servidor.\ De esta forma, únicamente se puede acceder al servidor
	mediante la
	clave privada del cliente.

	\item \textbf{Configuración del servidor web}: Se ha limitado la redirección de puertos desde el router hasta el
	servidor web, de forma que únicamente se puede acceder a los siguientes puertos:

	\begin{itemize}
		\item \boldFont{22}: puerto de ssh.
		\item \boldFont{80}: puerto de http.
		\item \boldFont{4040}: puerto de la aplicación en entorno de producción.
		\item \boldFont{4041}: puerto de la base de datos de la aplicación.
		\item \boldFont{4042}: puerto del servicio NSFW\@.
		\item \boldFont{4046}: puerto del panel de administración de Grafana.
	\end{itemize}

	Además, se ha configurado el servidor web para que únicamente se pueda acceder mediante el protocolo
	https, y no mediante http, haciendo que los usuarios sean redirigidos automáticamente a https.

	\item \textbf{Configuración de la base de datos}: La base de datos solo es accesible desde el propio servidor web,
	por lo que el acceso está restringido a la red interna.\ Para poder conectarse a la base de datos desde el
	exterior, se debe utilizar la clave ssh del servidor web, de forma que se accede a este y, desde ahí, se
	puede acceder a la base de datos.
\end{itemize}
