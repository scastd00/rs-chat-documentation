\chaptr{Conclusiones}{conclusiones}

En este capítulo se incluyen las conclusiones obtenidas tras el desarrollo del proyecto, así como una valoración
personal sobre el resultado final, las futuras ampliaciones y problemas encontrados.

Los \boldFont{objetivos} que se han propuesto en la sección~\ref{sec:objectives} \boldFont{se han cumplido} en su
totalidad dentro del plazo estimado.\ Se ha
conseguido desarrollar la aplicación web propuesta, permitiendo a los usuarios interactuar en tiempo real mediante
mensajes de todo tipo, incluyendo el filtrado de imágenes inapropiadas.

Algunas de las \boldFont{ampliaciones} que se han considerado implementar en un futuro son las siguientes:

\begin{itemize}
	\item \boldFont{Más tipos de mensajes}: se podrían añadir más tipos de mensajes, como por ejemplo, mensajes de
	ubicación, mensajes de voz, etc.
	\item \boldFont{Filtros para otro tipo de mensajes}: integrar los filtros existentes para mensajes de vídeo, audio
	u otros.
	\item \boldFont{Reenvío de mensajes a otros grupos}: permitir a los usuarios reenviar mensajes a otros grupos.
\end{itemize}

En cuanto a los \boldFont{problemas} encontrados, el más significativo ha sido el despliegue de la aplicación en
producción.\ Al inicio del proyecto, se desplegó en dos clusters con el plan gratuito de Heroku (uno
para el frontend y otro para el backend), pero se tuvo que retirar la aplicación de esta plataforma
debido a que este servicio no es gratuito desde el 28 de noviembre de 2022.
Actualmente, el frontend está desplegado en Vercel y en cuanto al backend, se consideraron varias opciones:

\begin{itemize}
	\item Despliegue en \textit{Platform.sh}: plataforma de despliegue de aplicaciones en la nube.
	\item Despliegue en un servidor propio.
\end{itemize}

La primera opción no ha sido posible, ya que, tras una reunión con el encargado de \textit{Cloud Services} de la
plataforma, no se ha podido conseguir una cuenta gratuita para el despliegue de la aplicación durante su desarrollo.
Por lo tanto, se ha optado por la segunda opción, comprando un pequeño ordenador personal y desplegando el backend en
él.\ De esta manera, también se abarca el campo de la administración de sistemas y redes, ya que se ha tenido que
configurar el servidor para que la aplicación funcione correctamente y se pueda acceder a ella desde cualquier lugar
con conexión a Internet.
