\usepackage{glossaries}\newglossaryentry{Frontend}{
	name={Frontend},
	description={Es la parte de la aplicación que se ejecuta en el navegador del usuario}
}

\newglossaryentry{Backend}{
	name={Backend},
	description={Es la parte de la aplicación que se ejecuta en el servidor}
}

\newglossaryentry{Cluster}{
	name={Cluster},
	description={Es un conjunto de servidores que trabajan juntos para realizar una tarea}
}

\newglossaryentry{Instanciar}{
	name={Instanciar},
	description={Es la acción de crear un objeto en memoria principal}
}

\newglossaryentry{Script}{
	name={Script},
	description={Es un programa que se ejecuta en un intérprete}
}

\newglossaryentry{Docker}{
	name={Docker},
	description={Es una herramienta que permite crear contenedores que ejecutan servicios}
}

\newglossaryentry{Docker Compose}{
	name={Docker Compose},
	description={Es una herramienta que permite definir y ejecutar contenedores Docker}
}

\newglossaryentry{Dockerfile}{
	name={Dockerfile},
	description={Es un fichero que contiene las instrucciones para crear una imagen Docker}
}

\newacronym{JSON}{JSON}{JavaScript Object Notation}
\newacronym{HTTP}{HTTP}{HyperText Transfer Protocol}
\newacronym{IDE}{IDE}{Integrated Development Environment (Entorno de desarrollo integrado)}
